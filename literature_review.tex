\chapter{Literature Review}
The primary objective of this paper is to propose a telemedicine system for diagnostic assistance in dermatoscopic, pharyngoscopic and otoscopic scans. Broadly, telemedicine is defined as the use of electronic information and communications technologies to provide and support healthcare when distance separates the participants. ~\cite{field} This encompasses a range of tools and platforms, from simple telephone consultations to advanced video conferencing and remote monitoring devices, allowing healthcare providers to evaluate, diagnose, and treat patients without the need for an in-person visit. The World Health Organization (WHO) further describes telemedicine as the delivery of healthcare services by all healthcare professionals using information and communication technologies for the exchange of valid information for diagnosis, treatment, prevention, research, and continuing education, all in the interest of advancing the health of individuals and their communities.~\cite{who} \par

The rise of telemedicine has been driven by several factors:

\begin{itemize}
    \item The need to provide healthcare in rural and underserved areas, as mentioned in the motivation of this paper.
    \item Growing consumer demand for convenience and timely care.
    \item Technical advancements in internet connectivity, mobile devices, and secure communications.
    \item The COVID-19 pandemic, which accelerated the adoption of telemedicine as a tool for delivering healthcare while minimizing infection risk.~\cite{who}
\end{itemize} 

The growth of AI across all fields influenced by software cannot be denied. It is changing telemedicine by improving efficiency, accuracy, and reach of remote healthcare services.~\cite{anto} Key applications of AI in telemedicine include Virtual Triage (AI analyzes patient symptoms and data to prioritize cases based on urgency, ensuring timely care and optimizing resource allocation), Remote Patient Monitoring (AI-powered devices and wearables collect and analyze real-time health data (e.g., heart rate, blood pressure, glucose levels), enabling proactive interventions and personalized care plans), Medical Imaging Analysis (AI systems assist clinicians by rapidly analyzing medical images (X-rays, MRIs, CT scans), improving diagnostic accuracy and accelerating treatment decisions) and predictive analytics (AI models analyze patient data to identify potential health risks, enabling early intervention and better disease management).~\cite{leeway} Our solution focuses on Medical Imaging Analysis of 3 specialized scans. \par

In our solution, we target 3 scans: dermatoscopy, pharyngoscopy and otoscopy.

\section{Pharyngoscopy}

Pharyngoscopy is a medical examination technique used to visualize the pharynx (throat) and adjacent structures. Clinically, this is most commonly performed by a physician shining a bright light into the patient`s mouth, sometimes using a small mirror or a fiberoptic scope to inspect the throat.~\cite{onto} This procedure is non-invasive and typically performed in an outpatient setting. \par


\
