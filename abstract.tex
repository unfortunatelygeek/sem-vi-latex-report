\begin{abstract}
India faces a severe shortage of healthcare specialists, particularly in ENT and dermatology. As of March 2023, over 23,000 sanctioned medical posts remained vacant across the country, with more than 9,000 vacancies in primary health centers (PHCs) alone. According to the National Health Profile 2019, India has only about 1 doctor per 1,457 people, far below the WHO recommendation of 1:1,000. The shortage is even more acute in rural areas, where approximately 70\% of India's population resides. Only about 12,000 in India for a population of 1.4 billion are ENT specialists (~1:116,000 ratio) Approximately 11,000 for 1.4 billion people are dermatologists (~1:127,000 ratio). Over 60\% of India's specialists practice in urban areas serving only 30\% of the population. 
Rural patients often travel long distances for basic diagnostic services. Average out-of-pocket expenditure for a single specialist consultation including travel can exceed INR 2,000-3,000. Loss of daily wages during travel further impacts economically vulnerable populations. Studies show that delayed diagnosis due to access barriers leads to more severe complications and higher treatment costs.
To mitigate the effects of this, we propose a smart telemedicine system that enables dermatoscopic, pharyngoscopic, and otoscopic imaging with a portable device consisting of a camera and microcontroller. The images captured are uploaded through a mobile app to create detailed, AI-enhanced diagnostic reports. Being designed for use in remote or underserved regions, the system equips primary care providers with sophisticated diagnostic tools improving healthcare delivery without significant infrastructure.
\end{abstract}
