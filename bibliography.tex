\begin{thebibliography}{}

% Literature Review

\bibitem{field}
~Field, M. J. (1996). Introduction and background. Telemedicine - NCBI Bookshelf. https://www.ncbi.nlm.nih.gov/books/NBK45440/
\bibitem{who}
~Dasgupta, A., \& Deb, S. (2008). Telemedicine: A new horizon in public health in India. Indian Journal of Community Medicine, 33(1), 3. https://doi.org/10.4103/0970-0218.39234
\bibitem{anto}
~Antoniuk, A. (2025, May 6). AI in Telemedicine: Challenges, Use Cases \& Implementation - TATEEDA | GLOBAL. TATEEDA | GLOBAL - Full cycle custom software development services and outsourcing in the USA and Ukraine. https://tateeda.com/blog/ai-in-telemedicine-use-cases. Retrieved on 12-05-2025, 17:46 IST (GMT +5:30)
\bibitem{leeway}
~Takyar, A. (2023, December 15). AI in telemedicine. LeewayHertz - AI Development Company. https://www.leewayhertz.com/ai-in-telemedicine/
\bibitem{tele}
~Mudur, G. (2024, September 10). 23,000 sanctioned medical posts vacant: Report flags rural-urban government doctor shortage. The Telegraph. https://www.telegraphindia.com/india/report-flags-rural-urban-government-doctor-shortage-23000-sanctioned-medical-posts-vacant/cid/2046918. Retrieved on 12-05-2025, 17:46 IST (GMT +5:30)
\bibitem{pib}
~Update on ratio of patients and doctors nurses. (n.d.). https://www.pib.gov.in/PressReleasePage.aspx?PRID=1985423
\bibitem{ndtv}
Press Trust of India. (2024, February 9). Doctor-Population ratio in country at 1:834, better than WHO standards: Health Minister. www.ndtv.com. https://www.ndtv.com/india-news/doctor-population-ratio-in-country-at-1-834-better-than-who-standards-health-minister-5025731. Retrieved on 12-05-2025, 17:46 IST (GMT +5:30)
\bibitem{swarno}
Swarno. (2025, March 8). Patient-to-Doctor ratio: a real challenge in maintaining quality healthcare. JMN Medical College. https://jmnmedicalcollege.org.in/patient-to-doctor-ratio-a-real-challenge-in-maintaining-quality-healthcare/ Retrieved on 12-05-2025, 17:46 IST (GMT +5:30)
\bibitem{dutta}
Dutta, S. S.(2025, March 3). Rural India has an 80\% shortfall of specialist doctors. MP, Gujarat, Tamil Nadu worst off. ThePrint. https://theprint.in/health/rural-india-has-an-80-shortfall-of-specialist-doctors-mp-gujarat-tamil-nadu-worst-off/2259874/ Retrieved on 12-05-2025, 17:46 IST (GMT +5:30)
\bibitem{bw}
National Doctor’s Day: Severe shortage of specialist doctors, rural areas most affected - BW Healthcare World. (n.d.). BW Health. https://www.bwhealthcareworld.com/article/national-doctor\%E2\%80\%99s-day-severe-shortage-of-specialist-doctors-rural-areas-most-affected-482599. Retrieved on 12-05-2025, 17:46 IST (GMT +5:30)

%Section: Dermatoscopy

\bibitem{oxio}
openoximetry.org/. (2024, September 13). Skin color quantification - OpenOximetry. OpenOximetry. https://openoximetry.org/skin-color-quantification/
\bibitem{book}
Schanda, J. (2007). Colorimetry: Understanding the CIE System. John Wiley \& Sons.
\bibitem{chardon}
Chardon, A., Cretois, I., \& Hourseau, C. (1991). Skin colour typology and suntanning pathways. International Journal of Cosmetic Science, 13(4), 191–208. https://doi.org/10.1111/j.1467-2494.1991.tb00561.x
\bibitem{kinnie}
Kinyanjui, N.M., Odonga, T., Cintas, C., Codella, N.C., Panda, R., Sattigeri, P., Varshney, K.R.: Fairness of classifiers across skin tones in dermatology. In: Medical Image Computing and Computer-Assisted Intervention. pp. 320–329 (2020)
\bibitem{isic2018}
Codella, N., Rotemberg, V., Tschandl, P., Celebi, M.E., Dusza, S., Gutman, D., Helba, B., Kalloo, A., Liopyris, K., Marchetti, M., et al.: Skin lesion analysis toward melanoma detection 2018: A challenge hosted by the international skin imaging collaboration (ISIC). arXiv preprint arXiv:1902.03368 (2019) 2, 13
\bibitem{sun}
Sun, X., Yang, J., Sun, M., Wang, K.: A benchmark for automatic visual classification of clinical skin disease images. In: European Conference on Computer Vision. pp. 206–222. Springer (2016)
\bibitem{groh}
Groh, M., Harris, C., Soenksen, L., Lau, F., Han, R., Kim, A., Koochek, A., Badri, O.: Evaluating deep neural networks trained on clinical images in dermatology with the Fitzpatrick 17k dataset. In: Proceedings of the IEEE/CVF Conference on Computer Vision and Pattern Recognition. pp. 1820–1828 (2021) 
\bibitem{nov}
Nouveau, Stephanie; Agrawal, Divya1; Kohli, Malavika2; Bernerd, Francoise; Misra, Namita1; Nayak, Chitra Shivanand3,. Skin Hyperpigmentation in Indian Population: Insights and Best Practice. Indian Journal of Dermatology 61(5):p 487-495, Sep–Oct 2016. | DOI: 10.4103/0019-5154.190103
\bibitem{sarangi}
Sarangi S, Das K, Padhi T.(2023).Clinical and Dermoscopic Evaluation of Melasma in Men- An Observational Study at a Tertiary Health Care Centre in Western Odisha, India,J Clin of Diagn Res. 17(2), WC05-WC09. https://www.doi.org/10.7860/JCDR/2023/59969/17522
\bibitem{alip}
Alipour N, Burke T, Courtney J. Skin Type Diversity in Skin Lesion Datasets: A Review. Curr Dermatol Rep. 2024;13(3):198-210. doi: 10.1007/s13671-024-00440-0. Epub 2024 Aug 14. PMID: 39184010; PMCID: PMC11343783.
\bibitem{agg}
Aggarwal P., Performance of Artificial Intelligence Imaging Models in Detecting Dermatological Manifestations in Higher Fitzpatrick Skin Color Classifications. JMIR Dermatol. 2021 Oct 12;4(2):e31697. doi: 10.2196/31697. PMID: 37632853; PMCID: PMC10334948.
\bibitem{mok}
Global burden of diseases, injuries, and risk factors for young people's health during 1990–2013: a systematic analysis for the Global Burden of Disease Study 2013
Mokdad, Ali H et al.
The Lancet, Volume 387, Issue 10036, 2383 - 2401
\bibitem{fitz}
Groh, M., Harris, C., Daneshjou, R., Badri, O., \& Koochek, A. (2022). Towards transparency in dermatology image datasets with skin tone annotations by experts, crowds, and an algorithm. Proceedings of the ACM on Human-Computer Interaction, 6(CSCW2), 1–26.
\bibitem{abhi}
Abhishek, K., Jain, A. \& Hamarneh, G. Investigating the Quality of DermaMNIST and Fitzpatrick17k Dermatological Image Datasets. Sci Data 12, 196 (2025). https://doi.org/10.1038/s41597-025-04382-5
\bibitem{pakzad}
Pakzad, A., Abhishek, K., Hamarneh, G., \& School of Computing Science, Simon Fraser University, Canada. CIRCLE: Color Invariant Representation Learning for Unbiased Classification of Skin Lesions. In School of Computing Science, Simon Fraser University, Canada. https://www2.cs.sfu.ca/~hamarneh/ecopy/eccv\_isic2022a.pdf

% Section: Pharyngoscopy

\bibitem{onto}
~Ontosight.Ai. (n.d.). Pharyngoscopy | Pharyngoscopy medical procedure overview. Ontosight.ai. https://ontosight.ai/glossary/term/pharyngoscopy-medical-procedure-overview--679d730638099fda3cfbca36 Retrieved on 12-05-2025, 17:46 IST \(GMT +5:30\)
\bibitem{kollara}
~Kollara, L., Perry, J. L., \& Hudson, S. (2015). Racial variations in velopharyngeal and craniometric morphology in Children: an imaging study. Journal of Speech Language and Hearing Research, 59(1), 27–38. https://doi.org/10.1044/2015\_jslhr-s-14-0236
\bibitem{choi}
~Choi, J. S., Yin, V., Wu, F., Bhatt, N. K., O’Dell, K., \& Johns, M. (2021). Utility of Telemedicine for Diagnosis and Management of Laryngology‐Related Complaints during COVID‐19. The Laryngoscope, 132(4), 831–837. https://doi.org/10.1002/lary.29838
\bibitem{selva}
Selvaraju, R. R., Cogswell, M., Das, A., Vedantam, R., Parikh, D., \& Batra, D. (2019). Grad-CAM: Visual Explanations from Deep Networks via Gradient-Based Localization. International Journal of Computer Vision, 128(2), 336–359. https://doi.org/10.1007/s11263-019-01228-7
\bibitem{nakajo}
~Nakajo, K., Ninomiya, Y., Kondo, H., Takeshita, N., Uchida, E., Aoyama, N., Inaba, A., Ikematsu, H., Shinozaki, T., Matsuura, K., Hayashi, R., Akimoto, T., \& Yano, T. (2023). Anatomical classification of pharyngeal and laryngeal endoscopic images using artificial intelligence. Head \& Neck, 45(6), 1549–1557. https://doi.org/10.1002/hed.27370
\bibitem{robo}
Pharyngitis Dataset Classification Dataset and Pre-Trained Model by Test 1. (2024, September 20). Roboflow. https://universe.roboflow.com/test-1-683jk/pharyngitis-dataset/dataset/4
\bibitem{encord2024}
Buhl, N. (2024, December 10). Training, validation, test split for machine learning datasets. https://encord.com/blog/train-val-test-split/
\bibitem{wilb}
Wilber, J. (n.d.). Train, test, and validation sets. MLU-Explain. https://mlu-explain.github.io/train-test-validation/

%Section: otoscopy



% ML

\bibitem{vit}
Dosovitskiy, A., Beyer, L., Kolesnikov, A., Weissenborn, D., Zhai, X., Unterthiner, T., Dehghani, M., Minderer, M., Heigold, G., Gelly, S., Uszkoreit, J., \& Houlsby, N. (2020, October 22). An Image is Worth 16x16 Words: Transformers for Image Recognition at Scale. arXiv.org. https://arxiv.org/abs/2010.11929

\bibitem{patch}
Wu, B., Xu, C., Dai, X., Wan, A., Zhang, P., Yan, Z., Tomizuka, M., Gonzalez, J., Keutzer, K., \& Vajda, P. (2020). Visual Transformers: Token-based Image Representation and Processing for Computer Vision. arXiv. https://arxiv.org/abs/2006.03677
\bibitem{imagenet}
Deng, J., Dong, W., Socher, R., Li, L.-J., Li, K., \& Fei-Fei, L. (2009). Imagenet: A large-scale hierarchical image database. In 2009 IEEE Conference on Computer Vision and Pattern Recognition (pp. 248–255). IEEE. https://doi.org/10.1109/CVPR.2009.5206848

\bibitem{shorten2019survey}
C.~Shorten and T.~M. Khoshgoftaar.
\newblock A survey on image data augmentation for deep learning.
\newblock {\em Journal of Big Data}, 6(1):1--48, 2019.

\bibitem{buslaev2018albumentations}
A.~Buslaev, A.~Parinov, E.~Khvedchenya, V.~Iglovikov, and A.~Kalinin.
\newblock Albumentations: fast and flexible image augmentations.
\newblock {\em arXiv preprint arXiv:1809.06839}, 2018.

\bibitem{cubuk2020randaugment}
E.~D. Cubuk, B.~Zoph, J.~Shlens, and Q.~V. Le.
\newblock RandAugment: Practical automated data augmentation with a reduced
  search space.
\newblock In {\em NeurIPS}, 2020.

\bibitem{wikipediaDataAugmentation}
Wikipedia contributors.
\newblock Data augmentation.
\newblock Retrieved May 2025, from https://en.wikipedia.org/wiki/Data\_augmentation.

\bibitem{wikipediaAlbumentations}
Wikipedia contributors.
\newblock Albumentations.
\newblock Retrieved May 2025, from https://en.wikipedia.org/wiki/Albumentations.

\bibitem{wikipediaBatchNorm}
Wikipedia contributors.
\newblock Batch normalization.
\newblock Retrieved May 2025

\bibitem{simard2003best}
P.~Y. Simard, D.~Steinkraus, and J.~C. Platt.
\newblock Best practices for convolutional neural networks applied to visual
  document analysis.
\newblock In {\em ICDAR}, 2003.
\bibitem{Valous2017}
N. Valous, “How can I remove the circles (different colours) from the dermoscopic image?”, ResearchGate (2017).

\bibitem{Pewton2022}
J. Pewton et al., “Dark Corner on Skin Lesion Image Dataset: Does It Matter?”, \emph{CVPR Workshops}, 2022.

\bibitem{Jaworek2013}
J. Jaworek‐Korjakowska and R. Tadeusiewicz, “Hair removal from dermoscopic color images”, \emph{Bio‐Algorithms and Med‐Systems}, vol. 9(2), pp. 53–58, 2013.

\bibitem{Khan2024}
M. Khan et al., “Classification of Skin Lesion with Hair and Artifacts Removal using Black‐hat Morphology and Total Variation”, ResearchGate, 2024.

\bibitem{SharpRazor2023}
A. F. Case et al., “Automatic removal of hair and ruler marks from dermoscopy images (SharpRazor)”, \emph{PMC}, 2023.

\bibitem{Buslaev2018}
A. Buslaev, A. Parinov, E. Khvedchenya, V. Iglovikov, and A.A. Kalinin, “Albumentations: Fast and Flexible Image Augmentations”, arXiv:1809.06839, 2018.

\bibitem{Buslaev2020}
A. Buslaev et al., “Albumentations: Fast and Flexible Image Augmentations”, \emph{Information}, vol. 11(2):125, 2020.

\bibitem{AlbumentationsWiki2024}
“Albumentations,” \emph{Wikipedia}, accessed 2024.

\bibitem{AlbumentationsDocs2024}
“Albumentations Documentation,” albumentations.ai/docs, accessed 2024.

\bibitem{AlbumentationsAPI2024}
“Module `albumentations.augmentations.transforms’”, Albumentations API, 2024.

\end{thebibliography}
