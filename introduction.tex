\chapter{Introduction}
The proposed system is a smart, telemedicine solution to aid in the development of Healthcare infrastructure in rural areas. The problem is defined below:
\begin{itemize}
\item Developing a smart imaging solution for medical diagnostics, specifically targeting otoscopy (ear examination), pharyngoscopy (throat examination), and dermatoscopy (skin examination). 
\item Integrate a camera with an STM microcontroller to capture high-resolution images
\item Perform image analysis for diagnostic assistance
\item Develop a user-friendly mobile or desktop app for medical professionals.
\end{itemize}
Rural India faces systemic healthcare barriers, including, but not limited to, limited diagnostic infrastructure, a shortage of trained specialists, and high out-of-pocket expenditures. With nearly 70\% of the population residing in rural regions, the availability of ENT specialists and dermatologists remains disproportionately low. approximately 1 per 116,000 and 1 per 127,000 individuals, respectively. Rural patients often experience delayed diagnosis and treatment for common but potentially serious ear, throat, and skin conditions.\par
To address these challenges, the integration of portable imaging devices, AI-driven diagnostic algorithms, and telemedicine platforms has emerged as a scalable solution. Recent advances in embedded systems, such as STM32-based devices running real-time operating systems like Zephyr OS, enable high-resolution image capture and wireless transmission. When combined with AI-based inference systems, these devices can assist non-specialist health workers in early identification of abnormalities from otoscopic, pharyngoscopic, and dermatoscopic scans.\par
This paper presents the design and implementation of a smart imaging system for rural diagnostics, incorporating embedded hardware, wireless data transfer, on-device preprocessing, and a React Native–based mobile application for real-time visualization and diagnostic assistance. The system aims to reduce diagnostic delays, improve specialist reach via telemedicine, and enhance primary care capabilities in resource-constrained settings. \par
\section{Motivation}
India faces a severe shortage of healthcare specialists, particularly in ENT and dermatology. As of March 2023, over 23,000 sanctioned medical posts remained vacant nationwide, with more than 9,000 vacancies specifically in primary health centres (PHCs).~\cite{tele} This shortage persists despite significant government efforts to expand medical education and infrastructure in recent years.~\cite{pib} The doctor-to-population ratio in India remains a subject of debate. While government sources report a ratio of 1:834-surpassing the World Health Organization (WHO) recommendation of 1:1,000-this figure includes both allopathic and AYUSH practitioners and assumes 80\% availability of registered doctors.~\cite{pib}\cite{ndtv} Independent analyses and the National Health Profile 2019 have previously cited a lower ratio of approximately 1:1,457, highlighting ongoing concerns about adequacy and distribution.~\cite{swarno} \par
The scarcity of specialists is particularly severe in fields such as otolaryngology (ENT) and dermatology. India has only about 12,000 ENT specialists for a population exceeding 1.4 billion, resulting in a ratio of roughly 1:116,000. Similarly, there are approximately 11,000 dermatologists, yielding a ratio of 1:127,000. These figures underscore the critical gaps in specialist care, particularly outside metropolitan areas.~\cite{dutta}\cite{bw}
An additional problem, on top of these shortages, is that over 60\% of India's specialists practice in urban areas, serving only 30\% of the country's population. \par
Closing this gap requires the use of technology-enabled innovations that allow experts to extend their services remotely, which can provide improved access to quality diagnostics and treatment across under-served areas.

\section{Objectives}
\begin{itemize}
\item Develop a smart imaging device for otoscopy, pharyngoscopy, and dermatoscopy.
\item Integrate a high-resolution camera with an STM32 microcontroller.
\item Use Zephyr OS for device firmware and Wi-Fi-based image transmission.
\item Implement image preprocessing (e.g., denoising, normalization).
\item Perform AI-based image analysis for diagnostic inference.
\item Build a cross-platform React Native app for image visualization and reporting.
\item Enable secure image transfer from device to app over Wi-Fi.
\item Implement patient data management and report generation in the app.
\end{itemize}
\section{Layout of the Report} 
A brief chapter by chapter overview is presented here.\\

Chapter 2: This chapter presents a detailed literature review of dermatoscopy, pharyngoscopy, and otoscopy, including datasets, imaging techniques, and current diagnostic solutions.\\

Chapter 3: A comprehensive market review is provided here, highlighting trends in Indian telemedicine, AI integration, and the relevance of teledermatology and ENT teleconsultation.\\

Chapter 4: This chapter describes the hardware components used in the system, including the camera and other things.\\

Chapter 5: This chapter outlines the technical implementation of the application, including system architecture, UI development, machine learning integration, data storage, and performance considerations.\\

Chapter 6: The chapter explains the core machine learning pipelines developed for pharyngoscopy and dermatoscopy using Vision Transformers and other deep learning models, along with the training, validation, and interpretability steps.\\

Chapter 7: Simulation and experimental results obtained from testing various components and machine learning models are presented and discussed in this chapter.\\

Chapter 8: This chapter provides the roadmap of the project, summarizing various phases such as literature review, simulation, hardware testing, and final deployment.\\

